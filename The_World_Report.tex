\documentclass[]{article}
\usepackage{lmodern}
\usepackage{amssymb,amsmath}
\usepackage{ifxetex,ifluatex}
\usepackage{fixltx2e} % provides \textsubscript
\ifnum 0\ifxetex 1\fi\ifluatex 1\fi=0 % if pdftex
  \usepackage[T1]{fontenc}
  \usepackage[utf8]{inputenc}
\else % if luatex or xelatex
  \ifxetex
    \usepackage{mathspec}
  \else
    \usepackage{fontspec}
  \fi
  \defaultfontfeatures{Ligatures=TeX,Scale=MatchLowercase}
\fi
% use upquote if available, for straight quotes in verbatim environments
\IfFileExists{upquote.sty}{\usepackage{upquote}}{}
% use microtype if available
\IfFileExists{microtype.sty}{%
\usepackage{microtype}
\UseMicrotypeSet[protrusion]{basicmath} % disable protrusion for tt fonts
}{}
\usepackage[margin=1in]{geometry}
\usepackage{hyperref}
\hypersetup{unicode=true,
            pdftitle={The World Report},
            pdfauthor={Us},
            pdfborder={0 0 0},
            breaklinks=true}
\urlstyle{same}  % don't use monospace font for urls
\usepackage{color}
\usepackage{fancyvrb}
\newcommand{\VerbBar}{|}
\newcommand{\VERB}{\Verb[commandchars=\\\{\}]}
\DefineVerbatimEnvironment{Highlighting}{Verbatim}{commandchars=\\\{\}}
% Add ',fontsize=\small' for more characters per line
\usepackage{framed}
\definecolor{shadecolor}{RGB}{248,248,248}
\newenvironment{Shaded}{\begin{snugshade}}{\end{snugshade}}
\newcommand{\KeywordTok}[1]{\textcolor[rgb]{0.13,0.29,0.53}{\textbf{#1}}}
\newcommand{\DataTypeTok}[1]{\textcolor[rgb]{0.13,0.29,0.53}{#1}}
\newcommand{\DecValTok}[1]{\textcolor[rgb]{0.00,0.00,0.81}{#1}}
\newcommand{\BaseNTok}[1]{\textcolor[rgb]{0.00,0.00,0.81}{#1}}
\newcommand{\FloatTok}[1]{\textcolor[rgb]{0.00,0.00,0.81}{#1}}
\newcommand{\ConstantTok}[1]{\textcolor[rgb]{0.00,0.00,0.00}{#1}}
\newcommand{\CharTok}[1]{\textcolor[rgb]{0.31,0.60,0.02}{#1}}
\newcommand{\SpecialCharTok}[1]{\textcolor[rgb]{0.00,0.00,0.00}{#1}}
\newcommand{\StringTok}[1]{\textcolor[rgb]{0.31,0.60,0.02}{#1}}
\newcommand{\VerbatimStringTok}[1]{\textcolor[rgb]{0.31,0.60,0.02}{#1}}
\newcommand{\SpecialStringTok}[1]{\textcolor[rgb]{0.31,0.60,0.02}{#1}}
\newcommand{\ImportTok}[1]{#1}
\newcommand{\CommentTok}[1]{\textcolor[rgb]{0.56,0.35,0.01}{\textit{#1}}}
\newcommand{\DocumentationTok}[1]{\textcolor[rgb]{0.56,0.35,0.01}{\textbf{\textit{#1}}}}
\newcommand{\AnnotationTok}[1]{\textcolor[rgb]{0.56,0.35,0.01}{\textbf{\textit{#1}}}}
\newcommand{\CommentVarTok}[1]{\textcolor[rgb]{0.56,0.35,0.01}{\textbf{\textit{#1}}}}
\newcommand{\OtherTok}[1]{\textcolor[rgb]{0.56,0.35,0.01}{#1}}
\newcommand{\FunctionTok}[1]{\textcolor[rgb]{0.00,0.00,0.00}{#1}}
\newcommand{\VariableTok}[1]{\textcolor[rgb]{0.00,0.00,0.00}{#1}}
\newcommand{\ControlFlowTok}[1]{\textcolor[rgb]{0.13,0.29,0.53}{\textbf{#1}}}
\newcommand{\OperatorTok}[1]{\textcolor[rgb]{0.81,0.36,0.00}{\textbf{#1}}}
\newcommand{\BuiltInTok}[1]{#1}
\newcommand{\ExtensionTok}[1]{#1}
\newcommand{\PreprocessorTok}[1]{\textcolor[rgb]{0.56,0.35,0.01}{\textit{#1}}}
\newcommand{\AttributeTok}[1]{\textcolor[rgb]{0.77,0.63,0.00}{#1}}
\newcommand{\RegionMarkerTok}[1]{#1}
\newcommand{\InformationTok}[1]{\textcolor[rgb]{0.56,0.35,0.01}{\textbf{\textit{#1}}}}
\newcommand{\WarningTok}[1]{\textcolor[rgb]{0.56,0.35,0.01}{\textbf{\textit{#1}}}}
\newcommand{\AlertTok}[1]{\textcolor[rgb]{0.94,0.16,0.16}{#1}}
\newcommand{\ErrorTok}[1]{\textcolor[rgb]{0.64,0.00,0.00}{\textbf{#1}}}
\newcommand{\NormalTok}[1]{#1}
\usepackage{graphicx,grffile}
\makeatletter
\def\maxwidth{\ifdim\Gin@nat@width>\linewidth\linewidth\else\Gin@nat@width\fi}
\def\maxheight{\ifdim\Gin@nat@height>\textheight\textheight\else\Gin@nat@height\fi}
\makeatother
% Scale images if necessary, so that they will not overflow the page
% margins by default, and it is still possible to overwrite the defaults
% using explicit options in \includegraphics[width, height, ...]{}
\setkeys{Gin}{width=\maxwidth,height=\maxheight,keepaspectratio}
\IfFileExists{parskip.sty}{%
\usepackage{parskip}
}{% else
\setlength{\parindent}{0pt}
\setlength{\parskip}{6pt plus 2pt minus 1pt}
}
\setlength{\emergencystretch}{3em}  % prevent overfull lines
\providecommand{\tightlist}{%
  \setlength{\itemsep}{0pt}\setlength{\parskip}{0pt}}
\setcounter{secnumdepth}{0}
% Redefines (sub)paragraphs to behave more like sections
\ifx\paragraph\undefined\else
\let\oldparagraph\paragraph
\renewcommand{\paragraph}[1]{\oldparagraph{#1}\mbox{}}
\fi
\ifx\subparagraph\undefined\else
\let\oldsubparagraph\subparagraph
\renewcommand{\subparagraph}[1]{\oldsubparagraph{#1}\mbox{}}
\fi

%%% Use protect on footnotes to avoid problems with footnotes in titles
\let\rmarkdownfootnote\footnote%
\def\footnote{\protect\rmarkdownfootnote}

%%% Change title format to be more compact
\usepackage{titling}

% Create subtitle command for use in maketitle
\newcommand{\subtitle}[1]{
  \posttitle{
    \begin{center}\large#1\end{center}
    }
}

\setlength{\droptitle}{-2em}

  \title{The World Report}
    \pretitle{\vspace{\droptitle}\centering\huge}
  \posttitle{\par}
    \author{Us}
    \preauthor{\centering\large\emph}
  \postauthor{\par}
      \predate{\centering\large\emph}
  \postdate{\par}
    \date{April 26, 2019}


\begin{document}
\maketitle

\section{I. Introduction:}\label{i.-introduction}

Team clearly described the dataset and clearly described the motivation
behind studying the data. Team provided scholarly citations or
quantitative facts to describe the motivation.

\section{II. Data Cleaning and Outlier
Visualization:}\label{ii.-data-cleaning-and-outlier-visualization}

Team clearly described their data cleaning and outlier removal process.
Team presented insightful visualizations motivating to do further
exploratory or confirmatory analysis.

\begin{Shaded}
\begin{Highlighting}[]
\CommentTok{#PART 1: Read csv, merge, clean and plot outliers.}
\KeywordTok{library}\NormalTok{(readr)}
\KeywordTok{library}\NormalTok{(readxl)}
\KeywordTok{library}\NormalTok{(dplyr)}
\KeywordTok{library}\NormalTok{(countrycode)}
\KeywordTok{library}\NormalTok{(car)}

\KeywordTok{source}\NormalTok{(}\StringTok{'Read_Clean.R'}\NormalTok{)}
\NormalTok{cleaned <-}\StringTok{ }\KeywordTok{Read_Clean}\NormalTok{()}
\end{Highlighting}
\end{Shaded}

\includegraphics{The_World_Report_files/figure-latex/unnamed-chunk-1-1.pdf}

\section{III. Dimension Reduction
Analysis:}\label{iii.-dimension-reduction-analysis}

Team applied dimension reduction analysis correctly and discussed the
motivation behind that. Also, they provided interesting insights into
the results.

\subsection{Part A: MDS}\label{part-a-mds}

\begin{itemize}
\item
  ALL VARIABLES INCLUDED -we can see Clusters -Asia: we can see is the
  most spread out and has the most outliers, spreads across Africa to
  Europe -Africa is the opposite of Europe -South America and North
  America are similar
\item
\end{itemize}

\begin{Shaded}
\begin{Highlighting}[]
\CommentTok{#PART 2: MDS}
\end{Highlighting}
\end{Shaded}

\begin{figure}
\centering
\includegraphics{img/3d.gif}
\caption{image}
\end{figure}

\begin{Shaded}
\begin{Highlighting}[]
\CommentTok{# PART 3: PCA}
\KeywordTok{library}\NormalTok{(pryr)}
\KeywordTok{library}\NormalTok{(ggbiplot) }\CommentTok{#if the library is not present use the code below}
\CommentTok{#library(devtools)}
\CommentTok{#install_github("vqv/ggbiplot")}
\KeywordTok{source}\NormalTok{(}\StringTok{'PCA.R'}\NormalTok{)}
\NormalTok{(PrinCompPlot <-}\StringTok{ }\KeywordTok{PCA}\NormalTok{(cleaned))}
\end{Highlighting}
\end{Shaded}

\begin{verbatim}
## [[1]]
\end{verbatim}

\includegraphics{The_World_Report_files/figure-latex/unnamed-chunk-3-1.pdf}

\begin{verbatim}
## 
## [[2]]
\end{verbatim}

\includegraphics{The_World_Report_files/figure-latex/unnamed-chunk-3-2.pdf}
FROM PCA excluded are: pop\_total, murder\_pp, armed\_pp,
urban\_pop\_tot, investment\_per\_of\_GDP it is because they spoil
correlation between variables and as such more PC would be needed to
explain the relation.

\subsection{PCA MEANING:}\label{pca-meaning}

\section{PC1: Developed countries, HIGH loaded in: phones, life exp.,
less corrup., internet
income}\label{pc1-developed-countries-high-loaded-in-phones-life-exp.-less-corrup.-internet-income}

\section{PC2: sex ratio is high, suicide
low}\label{pc2-sex-ratio-is-high-suicide-low}

\section{PC3: inequality is high}\label{pc3-inequality-is-high}

\begin{itemize}
\item
  ellipses to help visualize concentration of points. The size of
  ellipses is influenced by outlines.
\item
  on the right we have develope countires: EU, North America where the
  most correlated are: less corruption, internet, life exp., phones,
  income.
\item
  on the left have not as developed countries: Africa where Child
  mortality and children per woman and inequality is high. -Asia is the
  most diverse continent (especially in PC2). It also has the highest
  sex ration (means more mean per woman). Super low PC2 for sex ration
  (Quatar, UAE)
\item
  Central America \& South are in the middle on the plot
\item
  on PC1 \& PC3 explain better Africa. (it is because PC3 is mostly
  about inequality and development)
\end{itemize}

\begin{Shaded}
\begin{Highlighting}[]
\CommentTok{# PART 3: Hierarchical Clustering between Continents}
\KeywordTok{library}\NormalTok{(ape)}
\KeywordTok{source}\NormalTok{(}\StringTok{'cluster_continents.R'}\NormalTok{)}
\NormalTok{Cl_continents <-}\StringTok{ }\KeywordTok{cluster_continents}\NormalTok{(cleaned)}
\end{Highlighting}
\end{Shaded}

\includegraphics{The_World_Report_files/figure-latex/unnamed-chunk-4-1.pdf}
Include all variables

South, North and Europe are very similar. AND C America, Asia, Oceania
and Africa are similar. Interesting is Africa is clustered with Oceania
(with include Australia and NZ but also many small island which push
Oceania into level of Africa)

\begin{Shaded}
\begin{Highlighting}[]
\CommentTok{# PART 4: K-means & Model Based Clustering between Countries}
\KeywordTok{library}\NormalTok{(mclust)}
\KeywordTok{library}\NormalTok{(maptools)}
\KeywordTok{source}\NormalTok{(}\StringTok{'clusters_countries.R'}\NormalTok{)}
\NormalTok{Cl_countries <-}\StringTok{ }\KeywordTok{clusters_countries}\NormalTok{(cleaned)}
\end{Highlighting}
\end{Shaded}

\includegraphics{The_World_Report_files/figure-latex/unnamed-chunk-5-1.pdf}
\includegraphics{The_World_Report_files/figure-latex/unnamed-chunk-5-2.pdf}

\begin{itemize}
\item
  compare chi.square test -\textgreater{} dependency between groups and
  continents. Model based groups are more similar to continents.
\item
  model based (group7) difficult name (result for this group) pop\_total
  murder\_pp armed\_pp phones\_p100 children\_p\_woman life\_exp\_yrs
  suicide\_pp urban\_pop\_tot sex\_ratio\_p100 {[}1,{]} 239114394 0
  0.011 146.317 2.07 78.053 0 118882678 148.681 corruption\_CPI
  internet\_\%of\_pop child\_mort\_p1000 income\_per\_person
  investments\_per\_ofGDP gini {[}1,{]} 53.677 75.725 10.791 50579.31
  29.942 39.722
\item
  Developed countries are split into 3 groups.
\item
  poor countries are the same in both models
\item
  we lost ``crowded'' group from k-means. It transfoms into group 7
  which describe high Income, Sex Ratio, Population, phones
\end{itemize}

\begin{Shaded}
\begin{Highlighting}[]
\CommentTok{#PART 5: EFA}
\end{Highlighting}
\end{Shaded}

\begin{Shaded}
\begin{Highlighting}[]
\CommentTok{#PART 6: CFA}
\end{Highlighting}
\end{Shaded}


\end{document}
